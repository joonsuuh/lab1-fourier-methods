\documentclass[../main.tex]{subfiles}

\graphicspath{{../images/}}

\begin{document}
\pagestyle{fancy}
\lhead{Junseo Shin \& Jeremy Smith}
\rhead{Lab Notebook: Fourier Methods}
\chead{8/29/24}
% Day 1

\section*{8/29/24}
\addcontentsline{toc}{section}{8/29/24}

\paragraph*{The Timeline:}
\begin{itemize}
    \item 7 Class sessions
    \item Read ch 6 \& 15 for Signal to noise
    \item FM Com: ch 3, 5, 7, 18
    \item Acoustic vs. Fluxgate
\end{itemize}
\paragraph*{Familiarizing with Equipment (Chapter 0-2)}

Equipment list:
\begin{itemize}
    \item SR770 FFT Network Analyzer (main)
    \item Keysight 33500B Waveform Generator (Signal)
    \item Tektronix TDS 1012 (oscilloscope/scope)
    \item Teach Spin Fourier Methods Electronic Modules
\end{itemize}

\paragraph*{Observations:}
\begin{itemize}
    \item Triangle wave \& Square wave harmonics: [insert small table here]
    \item Sum of two sine waves(10 kHz and 20 kHz) can be easily identified in the frequency domain (SR770), but difficult to impossible in the oscilloscope (scope).
    \item Both 1 kHz vs 2
\end{itemize}

\paragraph*{frequency duration `uncertainty principle'}

\[\textrm{(frequency resolution achievable)} \cdot \textrm{(acquisition time required)} \geq  \textrm{a number}\]

Example: Given two frequencies, 50 Hz and 50.5 Hz

4ms acquisition time: 256 voltage samples per ms

\newpage
\section*{Signal recovery from under noise}

\end{document}